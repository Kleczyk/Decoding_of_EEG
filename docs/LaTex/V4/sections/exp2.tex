%! Author = daniel
%! Date = 20.01.2025

% Preamble
\documentclass[eeg_v4.tex]{subfiles}


% Document
\begin{document}




    \section{Eksperyment dla odseparowanych osób w zbiorze testowym i treningowym}

    Celem tego eksperymentu było zbadanie zdolności modelu do uogólniania w warunkach, w których osoby w zbiorze
    treningowym i testowym są odseparowane. W zbiorze treningowym uwzględniono dane od osób o numerach 1, 2, 8, 9, 14,
    17, 24 oraz 29, natomiast w zbiorze testowym znalazły się osoby o numerach 32 oraz 54. Pozostałe założenia
    eksperymentalne, takie jak wybór zadania, normalizacja danych, przestrzeń wyszukiwania hiperparametrów oraz
    struktura modelu i proces uczenia, były identyczne z opisanymi w sekcji \ref{sec:exp_non_separated}.

    \subsection{Wyniki eksperymentu}

    Wyniki tego eksperymentu wskazują na trudności modelu w uogólnianiu danych od nowych osób. Żaden z testowanych
    konfiguracji nie osiągnąła dokładności większej niż \(51\%\)
    , co świadczy o problemach z generalizacją w kontekście odseparowanych osób co obrzuje wykres równoległy
    hiperparametórw i dokłdaności \ref{fig:parallel_plot_separated}.

    \begin{figure}[h!]
        \centering
        \includegraphics[width=0.8\textwidth]{fig/parallel_plot_separated}
        \caption
        {Korelacje hiperparametrów z dokładnością na wykresie typu \textit{parallel plot} dla odseparowanych osób}
        \label{fig:parallel_plot_separated}
    \end{figure}

    \subsection{Wnioski}

    Dane od nowych osób w zbiorze testowym znacząco różniły się od danych w zbiorze treningowym. Modele nie były w
    stanie dostatecznie dobrze uogólnić, co skutkowało niską dokładnością klasyfikacji nie przekraczającą \(51\%\)
    . Wyniki te sugerują potrzebę
    dalszej analizy, szczególnie w kontekście różnic między indywidualnymi sygnałami EEG, oraz potencjalne zmiany w
    architekturze modelu lub strategii przetwarzania danych w przyszłych eksperymentach.




\end{document}