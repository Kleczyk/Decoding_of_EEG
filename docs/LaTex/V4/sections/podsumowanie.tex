%! Author = daniel
%! Date = 20.01.2025

% Preamble
\documentclass[eeg_v4.tex]{subfiles}

% Document
\begin{document}


    \section{Podsumowanie}
    \label{sec:podsumowanie}

    W niniejszej pracy przedstawiono wyniki eksperymentów dotyczących dekodowania intencji ruchowych na podstawie
    sygnałów
    EEG z wykorzystaniem modelu \texttt{LSTM + fully connected} (FC)
    . W części przeglądowej omówiono różnorodne podejścia do
    klasyfikacji sygnałów EEG, w tym architektury typu CNN czy CNN-LSTM \cite{boutarfaia2023,roots2020}, jednakże w ramach własnych badań
    skupiono się wyłącznie na modelu opartym na warstwie LSTM, do której wyjście zostało podłączone do w pełni
    połączonej
    warstwy końcowej (FC).

    \subsection*{Wyniki dla osób nieodseparowanych w zbiorze treningowym i testowym}
    Pierwszy eksperyment zakładał, że te same osoby (uczestnicy badania EEG) występują zarówno w zbiorze treningowym,
    jak i
    testowym. W tej konfiguracji – po dopasowaniu odpowiednich hiperparametrów (m.in. \textit{learning rate},
    \textit{hidden
    size}, \textit{dropout}
    i długości sekwencji) – uzyskano bardzo wysokie wyniki klasyfikacji. Najlepsze modele osiągnęły
    dokładność na poziomie ok. \(0{,}9833\), a pole pod krzywą ROC (AUC) przyjmowało wartości sięgające nawet
    \(0{,}99915\)
    na zbiorze testowym. Tak wysoka skuteczność wskazuje, że zastosowany model LSTM potrafi skutecznie wyodrębniać
    wzorce
    związane z realnym lub wyobrażonym ruchem kończyn w sytuacji, gdy dane od tych samych osób pojawiają się już w fazie
    treningu.

    \subsection*{Wyniki dla osób odseparowanych w zbiorze treningowym i testowym}
    Drugi eksperyment miał na celu zweryfikowanie, w jakim stopniu model jest w stanie uogólniać (generalizować) się na
    dane
    pochodzące od osób nieobecnych w zbiorze treningowym. W tym celu zbiór treningowy i testowy został podzielony tak,
    aby
    uczestnicy badania w zbiorze testowym byli całkowicie odseparowani od tych w zbiorze treningowym. Wyniki pokazały,
    iż
    model LSTM, w testowanej konfiguracji i bez dodatkowych technik adaptacyjnych, nie był w stanie przekroczyć
    dokładności
    ok. \(0{,}51\), co oznacza skuteczność zbliżoną do losowej.

    \subsection*{Wnioski główne}
    \begin{enumerate}
        \item \textbf{Skuteczność modelu w przypadku nieodseparowanych danych}
        Uzyskane wysokie wyniki (dokładność powyżej 0.98 oraz AUC bliskie 1.0) wskazują, że model LSTM znakomicie
        identyfikuje specyficzne wzorce EEG w sytuacji, gdy w zbiorze treningowym znajdują się dane od tych samych osób,
        które później występują w zbiorze testowym.

        \item \textbf{Brak uogólnienia na osoby spoza zbioru treningowego}
        W warunkach, gdy osoby ze zbioru testowego nie pojawiły się w zbiorze treningowym, dokładność modelu spada do
        poziomu
        ok. 0.5. Fakt ten wskazuje na duże zróżnicowanie sygnałów EEG między różnymi uczestnikami oraz konieczność
        opracowania lub zastosowania metod pozwalających na adaptację modelu do nowych użytkowników.

        \item \textbf{Perspektywy dla dalszych badań nad BCI}
        Zaprezentowane wyniki potwierdzają zasadność wykorzystania rekurencyjnych sieci LSTM w zadaniach typu
        \textit{Motor Imagery}. Jednocześnie podkreślają potrzebę wdrażania metod pozwalających na personalizację modelu
        bądź wykorzystanie transfer learningu, aby przełamać problem niskiej generalizacji na osoby nieobecne w fazie
        trenowania.
    \end{enumerate}

    \subsection*{Dalsze kierunki rozwoju}
    \begin{itemize}
        \item \textbf{Rozbudowa i zróżnicowanie zbioru danych:}
        Planuje się rozszerzyć liczbę uczestników badania EEG oraz włączyć dodatkowe zadania motoryczne lub
        wyobrażeniowe,
        co pozwoli na bardziej wiarygodną ocenę uogólniania modelu.

        \item \textbf{Personalizacja i transfer learning:}
        Jednym z kluczowych wyzwań w zastosowaniach \textit{Brain-Computer Interface} (BCI) jest dostosowanie modeli
        do indywidualnych cech użytkownika. Wprowadzenie metod \textit{transfer learningu} czy
        \textit{few-shot learningu}
        może znacząco poprawić wyniki uzyskiwane dla nowych osób.

        \item \textbf{Porównanie z innymi modelami:}
        Chociaż w niniejszej pracy skupiono się wyłącznie na modelu LSTM + FC, przyszłe prace mogą uwzględniać również
        architektury hybrydowe (np. CNN-LSTM), które w literaturze nierzadko wykazują wysoką skuteczność w zadaniach
        klasyfikacji sygnałów EEG \cite{boutarfaia2023,roots2020}.

        \item \textbf{Badanie stabilności i odporności na zakłócenia:}
        W praktycznych zastosowaniach BCI istotne jest, aby model pozostawał stabilny w obecności artefaktów (np. z
        ruchów gałek ocznych czy zakłóceń elektromagnetycznych). W związku z tym warto rozważyć mechanizmy dodatkowej
        filtracji sygnału i zaawansowane metody detekcji artefaktów.
    \end{itemize}

    Ostatecznie, uzyskane wyniki potwierdzają, iż sieci LSTM stanowią wartościowe narzędzie w dekodowaniu intencji
    ruchowych
    z EEG, zwłaszcza w warunkach, gdy dane uczące i testowe pochodzą od tych samych osób. Jednak w kontekście
    opracowywania
    uniwersalnych interfejsów mózg–komputer, zdolnych do natychmiastowego działania na nieznanych użytkownikach,
    konieczny
    jest dalszy rozwój metod adaptacyjnych i personalizowanych, które umożliwią znaczące zwiększenie ogólnej
    skuteczności
    klasyfikacji.


\end{document}