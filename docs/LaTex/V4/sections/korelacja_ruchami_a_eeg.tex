%! Author = daniel
%! Date = 20.01.2025

% Preamble
\documentclass[eeg_v4.tex]{subfiles}
% Document
\begin{document}
    \section{Korelacja między ruchami a sygnałami EEG}

    Rozpoznawanie intencji ruchowych z sygnałów EEG jest jednym z kluczowych obszarów badań w dziedzinie interfejsów
    mózg-komputer Brain-Computer Interface(BCI). W szczególności potencjały elektryczne mierzone w mózgu, które
    towarzyszą przygotowaniu i wykonaniu ruchów Movement-Related Cortical Potentials
    (MRCPs) oraz wzorce czasowo-częstotliwościowe odgrywają kluczową rolę w przewidywaniu kierunku ruchu.

    \subsection{Potencjały związane z ruchem}
    Jak pokazano w pracy \cite{wang2022}
    , MRCPs mogą być używane do przewidywania ruchów ręki na podstawie niskoczęstotliwościowych sygnałów EEG (0.01–4 Hz)
    Rysunek~\ref{fig:mrcp_plot}
    przedstawia średnie potencjały korowe \cite{Silva2020} (MRCPs)
    zarejestrowane w elektrodzie \textit{Cz}, która znajduje się w linii środkowej
    głowy (litera \textit{z} – Zero Line) i odpowiada środkowym obszarom pierwotnej kory ruchowej (\textit{C}
    – Central) . Wykres ilustruje zmiany potencjałów w czasie przygotowania i wykonania ruchu. Amplitudy potencjałów
    pozostawały stabilne w fazie przygotowania ruchu (od -1,5 do 0 s), natomiast podczas inicjacji ruchu zaobserwowano
    znaczną negatywną zmianę potencjału, co jest charakterystyczne dla aktywności związanej z wykonaniem ruchu.


    \begin{figure}[h]
        \centering
        \includegraphics[width=0.4\textwidth]{fig/mrcp_plot.png}
        \caption{Średnie potencjały korowe (MRCPs) w elektrodzie Cz dla kierunków ruchu prawej ręki (lewo i prawo)
            . Czas 0 s odnosi się do momentu inicjacji ruchu. Źródło: \cite{wang2022}}
        \label{fig:mrcp_plot}
    \end{figure}

    \subsection{Wzorce czasowo-częstotliwościowe}
    W pracy \cite{wang2022} zbadano również wzorce czasowo-częstotliwościowe. Rysunek~\ref{fig:time_frequency_plot}
    przedstawia widmo czasowo-częstotliwościowe, które ukazuje wzrost energii w niskich częstotliwościach (< 7 Hz) w
    czasie wykonywania ruchu. Wynik ten potwierdza, że główne modulacje mocy związane z ruchem są skoncentrowane w
    niskim zakresie częstotliwości.


    \begin{figure}[h]
        \centering
        \includegraphics[width=0.4\textwidth]{fig/time_frequency_plot.png}
        \caption
        {Widmo czasowo-częstotliwościowe sygnałów EEG w kanale Cz dla ruchu prawej ręki. Czas 0 s odnosi się do momentu
        inicjacji ruchu. Źródło: \cite{wang2022}}
        \label{fig:time_frequency_plot}
    \end{figure}

    \subsection{Dekodowanie kierunku ruchu}
    Dzięki zaawansowanym metodom, takim jak wykorzystanie nieliniowych dynamicznych parametrów MRCPs, w artykule
    \cite{wang2022} osiągnięto
    wysoką dokładność w dekodowaniu kierunków ruchu. Zastosowana metoda oparta na analizie sieci ESN (Echo State
    Networks) - sieci echo-stanu
    osiągnęła średnią dokładność dekodowania na poziomie $89.48\%$.

    Przeprowadzone badania potwierdzają, że sygnały EEG zawierają istotne informacje o ruchach, co uzasadnia możliwość
    tworzenia zaawansowanych modeli do dekodowania intencji ruchowych. Takie modele mogą być wykorzystane w praktycznych
    zastosowaniach, takich jak interfejsy BCI, systemy wspomagające rehabilitację, czy sterowanie protezami i
    egzoszkieletami.


\end{document}