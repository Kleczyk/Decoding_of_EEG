%! Author = daniel
%! Date = 20.01.2025

% Preamble
\documentclass[eeg_v4.tex]{subfiles}


% Document
\begin{document}

    \section*{Abstrakt}
    Celem niniejszej pracy jest opracowanie modelu rozpoznającego intencje ruchowe na podstawie sygnałów EEG,
    wykorzystującego różne konfiguracje rekurencyjnych sieci neuronowych (RNN), w tym w szczególności sieci LSTM (Long
    Short-Term Memory). Projekt bada możliwość precyzyjnego rozpoznania intencji ruchowych w warunkach offline oraz
    analizuje wymagania, które muszą zostać spełnione, aby osiągnąć wysoką dokładność tego procesu. Istotnym aspektem
    jest ocena możliwości uogólnienia modelu na dane pochodzące od osób spoza zbioru uczącego, co stanowi kluczowy
    element w kontekście praktycznych zastosowań. W pracy wykorzystano dane z \emph{EEG Motor Movement/Imagery Dataset}
    \cite{goldberger2000}
    , dostępne w zbiorze PhysioNet. Analiza opierała się na dwóch wybranych klasach ruchowych oraz jednej spoczynkowej.
    Przeprowadzone
    eksperymenty wykazały, że przy odpowiednim podziale próbek pomiędzy zbiór treningowy i walidacyjny, z zachowaniem
    tej
    samej liczebności klas, model osiągnął dokładność na poziomie 0.9833, a pole pod krzywą ROC uzyskało wartość 0.99915
    dla zbioru testowego zawierającego próbki od tych samych osób, które znajdowały się w zbiorze treningowym. Model
    oparty na sieci LSTM z warstwą wpełni połązconych neuronów (FC) fully-connected,
    który osiągnął najwyższą skuteczność, charakteryzował się następującymi parametrami: rozmiar warstwy ukrytej (hidden
    size) = 334, współczynnik uczenia (learning rate) = $3.587 \times 10^{-5}$
    , liczba warstw = 2, współczynnik dropout = 0.393, długość sekwencji = 86. Natomiast próba wyszkolenia modelu LSTM
    na danych, w których osoby z testowego zbioru danych były wyseparowane ze zbioru treningowego, zakończyła się
    niepowodzeniem – dokładność nigdy nie przekroczyła 0.5. Sugeruje to bardzo odmienny charakter danych dla różnych
    osób lub niewystarczającą zdolność modeli LSTM do uogólniania. Praca ma na celu przyczynienie się do rozwoju
    interfejsów mózg-komputer (BCI – Brain-Computer Interface), które mogą znaleźć zastosowanie w poprawie jakości życia
    osób z niepełnosprawnościami. Szczególny nacisk kładzie się na potencjalne wykorzystanie tego typu modeli do
    sterowania urządzeniami, takimi jak protezy, wyłącznie za pomocą myśli użytkownika. Wyniki badań mogą również
    dostarczyć cennych informacji na temat warunków niezbędnych do skutecznego wykorzystania sygnałów EEG w praktyce.


\end{document}