%! Author = daniel
%! Date = 20.01.2025

% Preamble
\documentclass[eeg_v4.tex]{subfiles}


% Document
\begin{document}
    \section{Analiza zbioru danych}

    Prezentowany zbiór danych EEG składa się z ponad 1500 jedno- i dwuminutowych nagrań, zebranych od 109 ochotników.
    Dane zostały zarejestrowane przy użyciu 64-kanałowego EEG zgodnie z systemem 10-10 wizualizacja rozkaładu \ref{fig:sensors} , wykorzystując system BCI2000.
    Uczestnicy wykonywali różne zadania motoryczne oraz ich wyobrażenia, takie jak otwieranie i zamykanie lewej lub
    prawej pięści, obu pięści czy obu stóp.

    \begin{figure}[h!]
        \centering
        \includegraphics[width=0.7\textwidth]{fig/sensors.png}
        \caption{Rozkład na głowie oraz nazwy sensorów EEG.}
        \label{fig:sensors}
    \end{figure}

    \subsection{Opis klas i ich ilość}

    Zbiór danych zawiera trzy główne klasy oznaczone jako $T0$, $T1$ i $T2$:

    \textbf{$T0$ (Stan spoczynku)}
    Odpowiada momentom, w których uczestnik jest w stanie relaksu, nie wykonuje żadnego ruchu ani nie wyobraża sobie
    jego wykonania.

    \textbf{$T1$ (Ruch lub wyobrażenie ruchu lewej pięści lub obu pięści)}
    \begin{itemize}
        \item Pomiary 3, 7, 11: Fizyczne otwarcie i zaciśnięcie lewej pięści.
        \item Pomiary 4, 8, 12: Wyobrażenie otwarcia i zaciśnięcia lewej pięści.
        \item Pomiary 5, 9, 13: Fizyczne otwarcie i skurczenie obydwóch stóp.
        \item Pomiary 6, 10, 14: Wyobrażenie otwarcia i skurczenia obydwóch stóp.
    \end{itemize}

    \textbf{$T2$ (Ruch lub wyobrażenie ruchu prawej pięści lub obu stóp)}
    \begin{itemize}
        \item Pomiary 3, 7, 11: Fizyczne otwarcie i zaciśnięcie prawej pięści.
        \item Pomiary 4, 8, 12: Wyobrażenie otwarcia i zaciśnięcia prawej pięści.
        \item Pomiary 5, 9, 13: Fizyczne otwarcie i zaciśnięcie obydwóch dłoni.
        \item Pomiary 6, 10, 14: Wyobrażenie otwarcia i zaciśnięcia obydwóch dłoni.
    \end{itemize}


    Klasy $T0$, $T1$, $T2$
    są definiowane różnie w zależności od numeru zadania. Poniżej przedstawiono warunki zadań oraz odpowiadające im
    numery pomiarów:

    \begin{enumerate}
        \item \textbf{Pomiar referencyjny, otwarte oczy}
        \item \textbf{Pomiar referencyjny, zamknięte oczy}
        \item \textbf{Zadanie 1} (otwieranie i zamykanie lewej lub prawej pięści)
        \item \textbf{Zadanie 2} (wyobrażenie otwierania i zamykania lewej lub prawej pięści)
        \item \textbf{Zadanie 3} (otwieranie i zamykanie obu pięści lub obu stóp)
        \item \textbf{Zadanie 4} (wyobrażenie otwierania i zamykania obu pięści lub obu stóp)
        \item \textbf{Zadanie 1}
        \item \textbf{Zadanie 2}
        \item \textbf{Zadanie 3}
        \item \textbf{Zadanie 4}
        \item \textbf{Zadanie 1}
        \item \textbf{Zadanie 2}
        \item \textbf{Zadanie 3}
        \item \textbf{Zadanie 4}
    \end{enumerate}

    Aby uprościć problem klasyfikacji oraz zgodnie z dobrymi praktykami rozpocząć od prostszego przypadku, zdecydowano
    się początkowo skupić na klasyfikacji trzech klas z zadania nr 1. Analizę przeprowadzono na danych pochodzących
    od 10 losowo wybranych osób o indeksach: {1, 2, 8, 9, 10, 13}. Takie podejście pozwala na opracowanie możliwie
    najlepszego modelu w ramach ograniczonego zakresu danych, który można następnie udoskonalać i rozszerzać na bardziej
    złożone przypadki oraz większe zbiory danych.

    \subsection{Charakterystyka sygnałów EEG}

    Sygnały EEG (elektroencefalograficzne) rejestrują aktywność elektryczną mózgu, charakteryzującą się niskimi
    amplitudami (rzędu mikrowoltów) oraz wysoką złożonością wynikającą z różnorodności stanów funkcjonalnych mózgu,
    takich jak czuwanie, sen czy aktywność poznawcza. Aktywność ta nie ogranicza się do jednego obszaru mózgu, lecz
    obejmuje różne regiony, które są aktywowane w różnym czasie i na różne sposoby. Utrudnia to jednoznaczne określenie,
    czy dany stan aktywności mózgu odpowiada konkretnej czynności.

    W celu przedstawienia charakterystyki sygnałów EEG wybrano kanał \textit{FCz}
    , ponieważ jego lokalizacja obejmuje obszar pierwotnej kory ruchowej (M1), która odpowiada m.in. za kontrolę ruchów
    kończyn \cite{Silva2020}
    . Należy jednak pamiętać, że jeden sensor nie jest w stanie w pełni odwzorować aktywności danego obszaru mózgu, co
    implikuje konieczność analizy danych z wielu sensorów jednocześnie, przez określony czas. Przykładowy zapis sygnału
    EEG o długości 20 sekund z sensora \textit{FCz} przedstawiono na Rysunku~\ref{fig:eeg_signal}. Sensor \textit{FCz}
    znajduje się w przedniej części linii środkowej głowy (litera \textit{F} – Frontal, \textit{C} – Central, \textit{z}
    – Zero Line) i w systemie rozmieszczenia elektrod 10-10 jest zlokalizowany najbardziej na środku obszaru pierwotnej
    kory ruchowej (M1). Jest on często wykorzystywany do monitorowania aktywności związanej z planowaniem ruchu oraz
    procesami poznawczymi. Widoczne są w nim fluktuacje o nieregularnym charakterze, typowe dla sygnału EEG.


    Dla porównania, sygnał EKG (elektrokardiograficzny) zaprezentowany na Rysunku~\ref{fig:ecg_signal}
    charakteryzuje się wyższymi amplitudami (rzędu miliwoltów) oraz wyraźną okresowością, która odzwierciedla cykl pracy
    serca. Podczas gdy sygnały EEG zawierają informacje o procesach neurofizjologicznych, sygnały EKG są bezpośrednio
    związane z mechaniką pracy mięśnia sercowego. Z tego względu sygnały EEG należą do najbardziej złożonych sygnałów
    biologicznych, co czyni ich analizę niezwykle trudnym zadaniem.


    \begin{figure}[h]
        \centering
        \includegraphics[width=0.8\textwidth]{fig/eeg_signal.png}
        \caption
        {Przykładowy sygnał EEG z kanału \textit{FCz} rejestrowany przez 20 sekund. z wydzielonymi klasami 0, 1, 2.}
        \label{fig:eeg_signal}
    \end{figure}

    \begin{figure}[h]
        \centering
        \includegraphics[width=0.8\textwidth]{fig/ekg_signal}
        \caption{Przykładowy sygnał ECG rejestrowany przez 20 sekund z z częstotliwością próbkowania 160 hz. przy użyciu
        bibloteki python neurokit2}
        \label{fig:ecg_signal}
    \end{figure}

    \subsection{Metryki statystyczne danych}
    \begin{table}[h]
        \centering
        \caption
        {Metryki statystyczne dla sygnału EEG z kanału \textit{FCz} (wartości w $\mu$V) oraz dla poszczególnych klas.}
        \label{tab:metrics}
        \begin{tabular}{|l|c|c|c|c|}
            \hline
            \textbf{Metryka} & \textbf{Cały sygnał} & \textbf{Klasa T0} & \textbf{Klasa T1} &
            \textbf{Klasa T2} \\
            \hline
            Liczba próbek                  & 118320   & 59280    & 30832     & 28208    \\
            \hline
            Maksimum                       & 390.0    & 390.0    & 376.0     & 359.0    \\
            \hline
            Minimum                        & -338.0   & -338.0   & -304.0    & -278.0   \\
            \hline
            Amplituda                      & 728.0    & 728.0    & 680.0     & 637.0    \\
            \hline
            Mediana                        & -4.0     & -5.0     & -1.0      & -5.0     \\
            \hline
            Maksimum wartości bezwzględnej & 390.0    & 390.0    & 376.0     & 359.0    \\
            \hline
            Średnia                        & -3.5714  & -5.2485  & 2.7959    & -7.0066  \\
            \hline
            Średnia wartości bezwzględnych & 45.1493  & 45.5618  & 45.1973   & 44.2300  \\
            \hline
            Wariancja                      & 4083.02  & 4134.47  & 4177.00   & 3810.15  \\
            \hline
            Odchylenie standardowe         & 63.8985  & 64.2999  & 64.6297   & 61.7264  \\
            \hline
            Energia                        & 484.6e6  & 246.7e6  & 129.0e6   & 108.9e6  \\
            \hline
            Energia sygnału wycentrowanego & 483.1e6  & 245.1e6  & 128.8e6   & 107.5e6  \\
            \hline
            Kurtoza                        & 2.6358   & 2.6572   & 2.7884    & 2.2596   \\
            \hline
            Skośność                       & 0.3120   & 0.2355   & 0.5454    & 0.1892   \\
            \hline
            Moment 5. rzędu                & 6452.7e6 & 5414.8e6 & 10084.6e6 & 4081.3e6 \\
            \hline
            Moment 6. rzędu                & 4290.7e9 & 4545.4e9 & 4760.0e9  & 2930.3e9 \\
            \hline
            Moment 7. rzędu                & 530.5e12 & 479.4e12 & 762.7e12  & 323.0e12 \\
            \hline
            Moment 8. rzędu                & 2.79e17  & 3.06e17  & 3.11e17   & 1.62e17  \\
            \hline
            Moment 9. rzędu                & 4.89e19  & 4.77e19  & 6.50e19   & 2.76e19  \\
            \hline
            Moment 10. rzędu               & 2.26e22  & 2.57e22  & 2.46e22   & 1.14e22  \\
            \hline
            Entropia Shannona              & 10.5300  & 9.8431   & 9.1462    & 9.1351   \\
            \hline
        \end{tabular}
    \end{table}


    Zestawienie metryk statystycznych \ref{tab:metrics}
    pozwala wstępnie zinterpretować różnice w aktywności EEG pomiędzy stanem
    spoczynku klasa T0 a dwoma klasami ruchu 1 i 2. W kontekście dekodowania ruchów z sygnału EEG warto zwrócić uwagę
    na kilka aspektów:

    \begin{itemize}
        \item \textbf{Średnia i mediana}
        \begin{itemize}
            \item Dla stanu spoczynku klasa T0 średnia wartość sygnału jest ujemna ok. \(-5.25\,\mu\mathrm{V}\)
            , a mediana również \(-5\,\mu\mathrm{V}\)
            . Wskazuje to, że potencjał w tej klasie zwykle utrzymuje się poniżej zera.
            \item W klasie T1 średnia jest dodatnia \(2.80\,\mu\mathrm{V}\), a mediana bliska zeru
            \(-1\,\mu\mathrm{V}\), co sugeruje częstsze wychylenia w kierunku wartości dodatnich.
            \item W klasie T2 średnia jest najbardziej ujemna \(-7.01\,\mu\mathrm{V}\)
            , a mediana wynosi \(-5\,\mu\mathrm{V}\), czyli wyraźnie utrzymuje się w dolnym zakresie rozkładu.
        \end{itemize}

        \item \textbf{Amplituda, wariancja i odchylenie standardowe}
        \begin{itemize}
            \item Największą amplitudę, czyli różnicę między maksimum a minimum, obserwujemy w klasie T0
            \(728\,\mu\mathrm{V}\) – czyli w stanie spoczynku. Klasy ruchowe mają nieco mniejszy rozrzut klasa T1:
            \(680\,\mu\mathrm{V}\), klasa T2: \(637\,\mu\mathrm{V}\).
            \item Wariancja i odchylenie standardowe są najwyższe dla klasy T1 odpowiednio \(4177\,\mu\mathrm{V}^2\) i
            \(64.63\,\mu\mathrm{V}\), a najniższe dla klasy T2 \(3810\,\mu\mathrm{V}^2\) i \(61.73\,\mu\mathrm{V}\)
            . Klasa T0 plasuje się między nimi \(4134.47\,\mu\mathrm{V}^2\) oraz \(64.30\,\mu\mathrm{V}\).
        \end{itemize}

        \item \textbf{Energia sygnału i energia wycentrowana}
        \begin{itemize}
            \item Największa energia całkowita i wycentrowana przypada na stan spoczynku klasa T0. Może to wskazywać na
            silną składową rytmów tła, takich jak rytm alfa lub beta.
            \item
            Klasy ruchowe T1 i T2 mają wyraźnie mniejszą energię całkowitą, co sugeruje krótsze epizody sygnału
            o większej amplitudzie lub inną charakterystykę częstotliwościową.
        \end{itemize}

        \item \textbf{Kurtoza i skośność}
        \begin{itemize}
            \itemWszystkie klasy mają kurtozę poniżej 3, co oznacza, że rozkłady są raczej „spłaszczone” w porównaniu do
            rozkładu normalnego.
            \item Skośność jest dodatnia dla wszystkich klas, ale najwyższa dla klasy T1 \(0.5454\)
            , co sugeruje, że rozkład próbek sygnału w ruchu klasa T1 ma wydłużony „ogon” w kierunku wartości
            dodatnich.
        \end{itemize}

        \item \textbf{Entropia Shannona}
        \begin{itemize}
            \item Najwyższą entropię wykazuje stan spoczynku klasa T0 \(9.8431\)
            , co może wskazywać na większą różnorodność rytmów i potencjałów.
            \item Klasy T1 i T2 mają niższe wartości entropii \(~9.15\)
            , co może świadczyć o bardziej specyficznej aktywności związanej z planowaniem/wykonywaniem ruchu.
        \end{itemize}

        \item \textbf{Implikacje dla dekodowania EEG}
        \begin{itemize}
            \item
            Różnice w średniej, rozkładach skośność i kurtoza, a także w zmienności sygnału wariancja, odchylenie
            standardowe sugerują, że poszczególne stany spoczynek vs. różne typy ruchu mogą mieć rozróżnialne cechy w
            dziedzinie czasowej.
            \item
            Entropia i momenty wyższych rzędów mogą pomóc w klasyfikacji, umożliwiając rozróżnienie klas przez algorytmy
            uczenia maszynowego.
        \end{itemize}
    \end{itemize}

    Podsumowując, metryki statystyczne wskazują, że sygnał w stanie spoczynku może być bardziej „rozproszony” i złożony
    wyższa entropia, największa amplituda, natomiast w klasach ruchu 1 i 2 zaobserwować można różnice w średniej,
    odchyleniu standardowym czy skośności. Analiza tych parametrów jest przydatna w projektowaniu algorytmów
    dekodujących intencję ruchu z sygnałów EEG.


%    \section{Infrastruktura Obliczeniowa i Środowisko Programistyczne}
%
%    \subsection{Specyfikacja sprzętu}
%    \begin{table}[h!]
%        \centering
%        \caption{Specyfikacja serwera i laptopa używanego w projekcie}
%        \label{tab:specyfikacja_sprzetu}
%        \resizebox{\columnwidth}{!}{%
%            \begin{tabular}{|l|l|}
%                \hline
%                \textbf{Komponent} & \textbf{Serwer}                                 \\ \hline
%                Procesor           & 32 x Intel Xeon Gold 6234 @ 3.30GHz (2 gniazda) \\
%                Pamięć RAM         & 256 GB                                          \\
%                Dysk SSD NVMe      & 1 TB                                            \\
%                Karta graficzna    & NVIDIA RTX 8000 (48 GB VRAM)                    \\
%                Dyski HDD          & 2 x 16 TB                                       \\
%                System operacyjny  & Debian GNU/Linux 12                             \\
%                \hline
%                \multicolumn{2}{|l|}{\textbf{Laptop: ROG Flow Z13 GZ301ZC}} \\ \hline
%                Procesor           & Intel i7-12700H (20 wątków) @ 4.60GHz           \\
%                Karty graficzne    & NVIDIA RTX 3050 Mobile, Intel Alder Lake-P      \\
%                Pamięć RAM         & 16 GB                                           \\
%                System operacyjny  & Pop!\_OS 22.04 LTS                              \\
%                \hline
%            \end{tabular}
%        }
%    \end{table}
%
%    \subsection{Środowisko obliczeniowe}
%    \begin{itemize}
%        \item \textbf{Wsparcie dla GPU:}
%        Kluczowe przy trenowaniu głębokich sieci neuronowych i przetwarzaniu dużych zbiorów danych.
%        \item \textbf{Praca rozproszona:} Możliwość efektywnego korzystania z wielu urządzeń.
%        \item \textbf{Optymalizacja hiperparametrów:} Automatyzacja zarządzania zasobami i optymalizacji.
%    \end{itemize}
%
%    \subsection{Narzędzia programistyczne}
%    \begin{itemize}
%        \item \textbf{Docker:} Konteneryzacja aplikacji zapewniająca izolację i skalowalność.
%        \item \textbf{Poetry:} Zarządzanie zależnościami w Pythonie.
%        \item \textbf{Git:} System kontroli wersji do śledzenia zmian w kodzie.
%        \item \textbf{Jupyter Notebook:}
%        Testowanie metod analizy danych i modeli z wykorzystaniem interaktywnego środowiska.
%    \end{itemize}
%
%    \subsection{Frameworki uczenia maszynowego}
%    \begin{itemize}
%        \item \textbf{PyTorch:} Dynamiczny graf obliczeniowy, wsparcie GPU.
%        \item \textbf{PyTorch Lightning:} Automatyzacja trenowania, zaawansowane logowanie i monitorowanie.
%        \item \textbf{Ray i RAITune:}
%        Zarządzanie pracą rozproszoną oraz optymalizacja hiperparametrów z wykorzystaniem metod bayesowskich.
%    \end{itemize}
%
%    \subsection{Przetwarzanie i wizualizacja danych}
%    \begin{itemize}
%        \item \textbf{Odczyt danych:}
%        Pliki .edf przetwarzane za pomocą biblioteki MNE, konwersja do formatu tabelarycznego (Pandas).
%        \item \textbf{Transformacja falkowa:} Użycie \texttt{PyWavelets}, falki \texttt{cgau4} do analizy sygnałów EEG.
%        \item \textbf{Monitorowanie uczenia:} Logowanie metryk i hiperparametrów za pomocą PyTorch Lightning i Weights
%        \\ Biases.
%    \end{itemize}
%
%    \begin{figure}[h!]
%        \centering
%        \includegraphics[width=0.3\textwidth]{fig/serwer_knml}
%        \caption{Serwer wykorzystywany w projekcie.}
%        \label{fig:serwer_knml}
%    \end{figure}
%
%    \begin{table}[h!]
%        \centering
%        \caption{Narzędzia do monitorowania procesu uczenia}
%        \label{tab:monitorowanie_uczenia}
%        \resizebox{\columnwidth}{!}{%
%            \begin{tabular}{|l|l|}
%                \hline
%                \textbf{Narzędzie} & \textbf{Funkcjonalność}                                     \\ \hline
%                PyTorch Lightning  & Logowanie metryk, hiperparametrów, integracja z TensorBoard \\
%                Weights \\ Biases        & Wizualizacja wielowymiarowych danych, analiza eksperymentów
%                \\
%                TensorBoard        & Podgląd metryk w czasie rzeczywistym                        \\
%                \hline
%            \end{tabular}
%        }
%    \end{table}


    \section{Podziału danych}
    Podział danych obejmuje dwa etapy: podział na sekwencje czasowe oraz na zbiory treningowy i walidacyjny. Najpierw dane dzielone są
    na mniejsze sekwencje czasowe (instancje) przy użyciu różnych metod segmentacji. Następnie próbki te są rozdzielane na
    zbiory treningowy i walidacyjny, z zachowaniem proporcji klas, co zapewnia reprezentatywność i poprawność oceny modelu.

    \subsection{Metody podziału danych na instancje}
    Tak jak w lietraturze \cite{boutarfaia2023,roots2020}
    zdecydowano się na podział danych na krótkie sekwencje czasowe o wybranej długości bez nakładania się, natomiast można rozważyć 3 różne podejścia poniżej
    przedstawiono opis oraz wizualizację każdej z tych metod.

    \paragraph{Pełne okna na całą klasę}
    W tej metodzie sekwencje rozpoczynają się równo z zmianą klasy są traktowane jako pojedyncze instancje. Liczba próbek w
    zbiorze treningowym i testowym odpowiada liczebności klas. Ta metoda pozwala na utrzymanie
    stałej struktury danych, co ułatwia interpretację wyników (problematyczne kiedy klasy mają różne długości). Przykład wizualizacji tej metody przedstawiono na rysunku
    \ref{fig:full_windows}.

    \begin{figure}[h!]
        \centering
        \includegraphics[width=0.5\textwidth,link=true]{fig/separate_fore_each_class}
        \caption{Podział na pełne okna (bez nakładania). (Rysunek własny)}
        \label{fig:full_windows}
    \end{figure}

    \paragraph{Okna o dowolnej długości bez nakładania się}
    Dane dzielone są na okna o określonej długości sekwencji (np. \texttt{sequence\_length} $=2 s$ ) kążde okno jest pojedynczą instancją
    oraz jest niezależne i następuje bezpośrednio po poprzednim. Zwiększenie liczby krótszych okien pozwala na
    uzyskanie większej liczby próbek w zbiorze treningowym, co może poprawić zdolności generalizacji modelu. Przykład
    wizualizacji tej metody przedstawiono na rysunku \ref{fig:non_overlapping}.

    \begin{figure}[h!]
        \centering
        \includegraphics[width=0.5\textwidth,link=true]{fig/separate_fore_seq_length}
        \caption{Okna o dowolnej długości (bez nakładania). (Rysunek własny)}
        \label{fig:non_overlapping}
    \end{figure}

    \paragraph{Okna o dowolnej długości z nakładaniem się}
    W tej metodzie okna o długości sekwencji \texttt{sequence\_length}
    nachodzą na siebie. Dla zobrazowania, przesunięcie
    $overlap = 50\%$
    powoduje, że każde kolejne okno rozpoczyna się w połowie poprzedniego. Dzięki temu uzyskuje się większą
    liczbę instancji, kosztem mniejszej różnorodności danych. Przykład
    wizualizacji tej metody przedstawiono na rysunku \ref{fig:overlapping}.

    \begin{figure}[h!]
        \centering
        \includegraphics[width=0.5\textwidth,link=true]{fig/overlappint_fore_seq_length}
        \caption{Okna o dowolnej długości (z nakładaniem 50\%). (Rysunek własny)}
        \label{fig:overlapping}
    \end{figure}

    \subsection{Podział zbioru danych na zbiory treningowy i walidacyjny}
    W ramach eksperymentu zdecydowano się na podział danych na dwa zbiory: treningowy i walidacyjny. Pomimo że w wielu
    przypadkach stosuje się dodatkowy zbiór walidacyjny, w tym badaniu ograniczono się do dwóch zbiorów w celu
    oszczędności czasu. Zbiór treningowy stanowił 80\% danych, a zbiór walidacyjny 20\%
    , przy zachowaniu proporcji klas, co zapewniło danych w obu zestawach.

    Do badań w tym eksperymencie wybrano metodę dzielenia na próbki z użyciem okien o dowolnej długości bez nakładania.
    Wybór ten podyktowany był możliwością elastycznego dopasowania długości sekwencji, co pozwala na dokładniejsze
    zbadanie wpływu tego parametru na wyniki modelu. Metoda z pełnymi oknami dla całych klas została odrzucona, gdyż nie
    umożliwia dostosowania długości sekwencji. Z kolei metoda z oknami o dowolnej
    długości z nakładaniem została wykluczona ze względu na generowanie dużej liczby próbek, co przy dostępnych zasobach
    obliczeniowych znacząco wydłużyłoby czas szkolenia modeli.


    \section{Przetwarzanie danych EEG do eksperymentu}

    Przetwarzanie sygnałów EEG przed szkoleniem modeli wymaga zastosowania odpowiednich technik wstępnej obróbki. W tej
    pracy skupiamy się na dwóch kluczowych etapach: normalizacji oraz transformacji falkowej.

    \subsection{Normalizacja sygnałów EEG}

    W celu zapewnienia spójności danych i poprawy efektywności modeli uczenia maszynowego, sygnały EEG zostały poddane
    normalizacji min-max. Normalizacja ta jest wykonywana osobno dla każdego kanału w każdym eksperymencie, co pozwala
    na zachowanie względnych różnic między wartościami sygnału w obrębie jednego kanału, jednocześnie eliminując różnice
    w amplitudzie między kanałami i eksperymentami.

    Wzór na normalizację min-max dla każdego kanału w danym eksperymencie przedstawiono poniżej:

    \[
        X_{norm} = \frac{X - X_{min}}{X_{max} - X_{min}}
    \]

    gdzie:
    \begin{description}
        \item \(X\) – oryginalny sygnał EEG,
        \item \(X_{min}\) – minimalna wartość sygnału dla danego kanału w eksperymencie,
        \item \(X_{max}\) – maksymalna wartość sygnału dla danego kanału w eksperymencie,
        \item \(X_{norm}\) – znormalizowany sygnał EEG, przyjmujący wartości z zakresu \([0, 1]\).
    \end{description}

    Normalizacja min-max jest szczególnie przydatna w analizie sygnałów EEG, ponieważ zachowuje względne różnice między
    wartościami sygnału, jednocześnie ograniczając je do ustalonego zakresu. Przykłady zastosowania tego podejścia można
    znaleźć w literaturze, np. w \cite{roots2020}
    , gdzie wykazano, że normalizacja sygnałów EEG istotnie poprawia jakość klasyfikacji.


    \subsection{Transformacja falkowa}

    Transformacja falkowa (ang. \textit{Wavelet Transform}
    , WT) jest techniką analizy sygnałów, która umożliwia dekompozycję danych na składowe o różnej skali. Jest to
    szczególnie użyteczne w analizie sygnałów EEG, które charakteryzują się zmiennością zarówno w czasie, jak i w
    częstotliwości.

    \subsubsection{Wzór na ciągłą transformatę falkową}

    Ciągła transformata falkowa sygnału \( x(t) \) jest definiowana jako:

    \[
        W_x(a, b) = \frac{1}{\sqrt{|a|}} \int_{-\infty}^{\infty} x(t) \, \psi^*\left(\frac{t - b}{a}\right) \, dt
    \]

    gdzie:
    \begin{description}
        \item \( W_x(a, b) \) – współczynnik transformaty falkowej dla skali \( a \) i przesunięcia \( b \),
        \item \( x(t) \) – analizowany sygnał,
        \item \( \psi(t) \) – funkcja falki (tzw. falka matka),
        \item \( \psi^*(t) \) – sprzężenie zespolone falki \( \psi(t) \),
        \item \( a \) – parametr skali (rozciągnięcie lub skurczenie falki),
        \item \( b \) – parametr przesunięcia (lokalizacja czasowa),
        \item \( t \) – czas.
    \end{description}

    Wzór ten opisuje, jak sygnał \( x(t) \) jest korelowany z falką \( \psi(t) \)
    w różnych skalach i przesunięciach, co umożliwia analizę lokalnych cech sygnału w dziedzinie czas-częstotliwość.
    Szczegółowe omówienie transformaty falkowej oraz jej zastosowań można znaleźć w literaturze, np. w
    \cite{bibliotekanauki}
    , gdzie przedstawiono podstawy teoretyczne transformaty falkowej oraz przykłady jej zastosowań w analizie sygnałów.

    \subsubsection{Typ falki}

    W analizie sygnałów EEG często stosuje się różne rodzaje falek, w zależności od charakterystyki sygnału i celu
    analizy. Dwa popularne typy falek to falka Morleta oraz falki Daubechies.

    \paragraph{Falka Morleta}
    Falka Morleta \( \psi(t) \) jest sinusoidą modulowaną oknem Gaussa i jest zdefiniowana jako:
    \[
        \psi(t) = \exp\left(-\frac{t^2}{2}\right) \cdot \exp(j\omega t),
    \]
    gdzie:
    \begin{description}
        \item \( j \) – jednostka urojona (\( j^2 = -1 \)),
        \item \( \omega \) – częstotliwość centralna falki.
    \end{description}
    Falka Morleta jest szczególnie przydatna w analizie sygnałów EEG, ponieważ dobrze odwzorowuje składowe o charakterze
    oscylacyjnym, co jest typowe dla sygnałów mózgowych.

    \paragraph{Falki Daubechies}
    Falki Daubechies to rodzina falek ortogonalnych, które są szeroko stosowane w analizie sygnałów ze względu na ich
    dobre właściwości czasowo-częstotliwościowe. Falki te charakteryzują się zwartym nośnikiem (są niezerowe tylko w
    skończonym przedziale czasu) oraz różną liczbą momentów znikających, co pozwala na efektywną dekompozycję sygnałów.
    W tej pracy zastosowano falę Daubechies 4 (\textit{db4}
    ), która jest jedną z najczęściej wykorzystywanych falek w analizie EEG. Falka db4 dobrze odwzorowuje szybkie zmiany
    sygnałów, co czyni ją odpowiednią dla danych EEG.

    \subsubsection{Wzór na falkę Daubechies 4 (db4)}
    Falka Daubechies 4 jest definiowana przez swoje współczynniki skalujące i falkowe. Współczynniki skalujące dla falki
    db4 są następujące:

    \[
        \begin{aligned}
            h_0 &= \frac{1 + \sqrt{3}}{4\sqrt{2}}, \\
            h_1 &= \frac{3 + \sqrt{3}}{4\sqrt{2}}, \\
            h_2 &= \frac{3 - \sqrt{3}}{4\sqrt{2}}, \\
            h_3 &= \frac{1 - \sqrt{3}}{4\sqrt{2}},
        \end{aligned}
    \]

    gdzie \( h_0, h_1, h_2, h_3 \) to współczynniki skalujące, a współczynniki falkowe \( g_k \) są obliczane jako:

    \[
        g_k = (-1)^k h_{3-k} \quad \text{dla} \quad k = 0, 1, 2, 3.
    \]

    Funkcja skalująca \( \phi(t) \) i falkowa \( \psi(t) \) są definiowane rekurencyjnie poprzez te współczynniki:

    \[
        \phi(t) = \sqrt{2} \sum_{k=0}^{3} h_k \phi(2t - k),
    \]

    \[
        \psi(t) = \sqrt{2} \sum_{k=0}^{3} g_k \phi(2t - k).
    \]

    Te równania opisują, jak funkcja skalująca i falkowa są generowane na podstawie współczynników \( h_k \) i \( g_k \)
    \cite{daubechies1992}.

    \subsubsection{Zastosowanie falki Daubechies w analizie sygnałów}
    Falki Daubechies są szeroko stosowane w analizie sygnałów, szczególnie w kontekście dyskretnej transformacji
    falkowej (DWT). Są one wykorzystywane do dekompozycji sygnałów na składowe o różnej skali, co jest szczególnie
    przydatne w analizie sygnałów EEG, kompresji danych oraz usuwaniu szumów \cite{mallat1999}.

    \subsubsection{Zastosowanie transformacji falkowej}

    Transformacja falkowa jest wykorzystywana wyłącznie w modelach, które integrują splotowe sieci neuronowe (CNN). W
    artykule \cite{rajwal2023}
    wykazano, że transformacja falkowa w połączeniu z CNN znacząco poprawia dokładność klasyfikacji sygnałów EEG. CNN
    potrafią efektywnie uczyć się wzorców na podstawie cech czasowo-częstotliwościowych uzyskanych dzięki transformacji
    falkowej.

    \subsection{Podsumowanie}

    Zaproponowana metodologia przetwarzania danych EEG obejmuje normalizację danych dla każdego eksperymentu z osobna
    oraz transformację falkową z wykorzystaniem falki Daubechies 4. Transformacja falkowa umożliwia efektywne wydobycie
    cech czasowo-częstotliwościowych, co jest kluczowe w przypadku modeli opartych na sieciach spolotowych. Dzięki
    takiemu podejściu możliwe jest uzyskanie wysokiej dokładności klasyfikacji sygnałów EEG w ramach eksperymentów
    naukowych.

\end{document}