%%%%%%%%%%%%%%%%%%%%%%%%%%%%%%%%%%%%%%%%%%%%%%%%%%%%%%%%%%%%%%%%%
%%% %
%%% % weiiszablon.tex
%%% % The Faculty of Electrical and Computer Engineering
%%% % Rzeszow University Of Technology diploma thesis Template
%%% % Szablon pracy dyplomowej Wydziału Elektrotechniki 
%%% % i Informatyki PRz
%%% % June, 2015
%%%%%%%%%%%%%%%%%%%%%%%%%%%%%%%%%%%%%%%%%%%%%%%%%%%%%%%%%%%%%%%%%


\documentclass[12pt,twoside]{article}

\usepackage{weiiszablon}
\usepackage{url}
\author{Daniel Kleczyński}

% np. EF-123456, EN-654321, ...
\studentID{XX-??????}

\title{Wykorzystanie rekurencyjnych sieci neuronowych w
dekodowaniu intencji ruchowych na podstawie sygnałów EEG}
\titleEN{Temat pracy po angielsku}


%%% wybierz rodzaj pracy wpisując jeden z poniższych numerów: ...
% 1 = inżynierska	% BSc
% 2 = magisterska	% MSc
% 3 = doktorska		% PhD
%%% na miejsce zera w linijce poniżej
\newcommand{\rodzajPracyNo}{0}


%%% promotor
\supervisor{ prof. dr hab. inż. Jacek Kluska}
%% przykład: dr hab. inż. Józef Nowak, prof. PRz

%%% promotor ze stopniami naukowymi po angielsku
\supervisorEN{(academic degree) Imię i nazwisko opiekuna}

\abstract{Treść streszczenia po polsku}
\abstractEN{Treść streszczenia po angielsku}

\begin{document}

% strona tytułowa
\maketitle

\blankpage

% spis treści
\tableofcontents

\clearpage
\blankpage


\begin{abstract}

Projekt ma na celu stworzenie modelu rekurencyjnych sieci neuronowych (RNN), który może być wykorzystany do dekodowania dziewięciu różnych intencji ruchowych na podstawie sygnałów EEG w czasie rzeczywistym dla interfejsów mózg-komputer. Analizowane są różne architektury sieci LSTM, w tym ilość warstw, dwukierunkowość, oraz wielkość warstwy ukrytej. Badania obejmują także transformację falkową, z uwzględnieniem typu falki, długości sekwencji i rozdzielczości transformacji. Model jest trenowany na specyficznym zbiorze danych "EG Motor Movement/Imagery Dataset" \ref{physionet_eegmmidb}, co pozwala na dokładne dostosowanie i optymalizację modelu. W ramach projektu rozwijane jest środowisko do testowania i szkolenia modeli, umożliwiające precyzyjną regulację hiperparametrów, takich jak wielkość wsadu, współczynnik uczenia, wielkość warstwy ukrytej oraz długość sekwencji. Wyniki mają na celu nie tylko opracowanie efektywnego modelu,  ale również przyczynienie się do rozwoju technologii interfejsów mózg-komputer, zwiększając ich funkcjonalność i efektywność. Wyniki te mogą pomóc osobom niepełnosprawnym, ułatwiając komunikację i interakcję ze światem zewnętrznym, oraz przyspieszyć rozwój systemów BCI, otwierając nowe możliwości dla technologii wspomagających. Co ważne, opierając się na możliwościach technologii wspomagających bez konieczności ingerencji w ciało ludzkie, EEG stanowi zewnętrzne urządzenie, co dodatkowo zwiększa dostępność i bezpieczeństwo stosowania tych rozwiązań.
\end{abstract}

\clearpage
\section{Wstęp}
Wprowadzenie do projektu, znaczenie zbudowanego środowiska dla celów badawczych.

\section{Opis sprzętu}
\subsection{Specyfikacja serwera}
Szczegółowy opis serwera, w tym informacje o GPU RTX 8000, procesorach Xeon, oraz 250 GB pamięci RAM.

\subsection{Dostęp i oprogramowanie}
Metody łączenia się z serwerem, system operacyjny, oprogramowanie zainstalowane i jego rola w przetwarzaniu danych.

\section{Zarządzanie środowiskiem i automatyzacja}
Zarządzanie środowiskiem projektowym i automatyzacja procesów są kluczowe dla zapewnienia efektywności i powtarzalności w pracy nad zaawansowanymi projektami inżynierskimi. W projekcie wykorzystano następujące narzędzia do automatyzacji i zarządzania środowiskiem:

\subsection{Docker}
Docker jest używany do konteneryzacji aplikacji i zależności, co umożliwia łatwą replikację środowiska na różnych maszynach bez konieczności ręcznej konfiguracji. Kontenery Docker zapewniają izolację, niezawodność i skalowalność aplikacji, umożliwiając jednoczesne uruchamianie wielu instancji modelu na różnych serwerach.

\subsection{Poetry}
Poetry jest wykorzystywane do zarządzania zależnościami Pythona i pakowaniem projektów. Ułatwia ono zarządzanie bibliotekami i zależnościami, zapewniając konsystencję wersji pakietów między środowiskami deweloperskim i produkcyjnym. Poetry także automatycznie tworzy wirtualne środowiska, co izoluje zależności projektu od globalnych instalacji Pythona.

\subsection{Git}
Git służy do wersjonowania kodu źródłowego, co jest niezbędne w pracy zespołowej i przy zarządzaniu zmianami w projekcie. Dzięki systemowi kontroli wersji, wszyscy uczestnicy projektu mogą efektywnie współpracować, śledzić wprowadzone zmiany i przywracać poprzednie wersje kodu, gdy jest to konieczne.

\subsection{Automatyzacja zadań}
Do automatyzacji rutynowych zadań, takich jak testy, kompilacja i wdrażanie aplikacji, stosowane są skrypty i narzędzia CI/CD (Continuous Integration/Continuous Deployment). Te procesy automatyzacji zwiększają produktywność i zmniejszają ryzyko błędów ludzkich.

Kombinacja tych narzędzi znacznie poprawia efektywność rozwoju projektu, minimalizuje możliwość wystąpienia błędów konfiguracyjnych i ułatwia skalowanie aplikacji na różnych platformach i środowiskach.


\section{Przetwarzanie i wizualizacja danych}
\subsection{Transformacja falkowa}
Opis stosowania transformacji falkowej do przetwarzania danych EEG.

\subsection{Wizualizacja danych}
Użycie narzędzia Biases do wizualizacji danych, przedstawienie, jak narzędzie to pomaga w analizie i prezentacji wyników.

\section{Środowisko obliczeniowe}
\subsection{Biblioteki uczenia maszynowego}
Opis stosowania PyTorch i Lightning do trenowania modeli.

\subsection{Ray i RAITune}
Wykorzystanie Ray do zarządzania rozproszonymi zasobami oraz RAITune do optymalizacji hiperparametrów.

\section{Podsumowanie}
Krótkie podsumowanie korzyści płynących z zastosowanych rozwiązań i wpływ na osiągane wyniki w projekcie.




\section*{Załączniki}
\addcontentsline{toc}{section}{Załączniki}

Według potrzeb zawarte i uporządkowane uzupełnienie pracy o dowolny materiał źródłowy (wydruk programu komputerowego, dokumentacja kons\-truk\-cyj\-no-\-tech\-no\-lo\-gicz\-na, konstrukcja modelu -- makiety -- urządzenia, instrukcja obsługi urządzenia lub stanowiska laboratoryjnego, zestawienie wyników pomiarów i obliczeń, informacyjne materiały katalogowe itp.).


\clearpage

\addcontentsline{toc}{section}{Literatura}

\begin{thebibliography}{99} 
	
	\bibitem{physionet_eegmmidb} Schalk, G. et al.: EEG Motor Movement/Imagery Dataset. PhysioNet. Dostępne na: \url{https://physionet.org/content/eegmmidb/1.0.0/} (dostęp: 10.06.2024).
\end{thebibliography}


\clearpage

\makesummary

\end{document}
